\documentclass[12pt]{article}
\usepackage{geometry}
\usepackage{float}
\usepackage{fancyhdr}
\usepackage{graphicx}
\usepackage{titling}
\usepackage{hyperref}
\usepackage[hyphens]{url}
\usepackage[normalem]{ulem}

\geometry{a4paper, margin=1in}
\Urlmuskip=0mu plus 1mu

\title{JPL - Moon Augmented Reality \\ Software Requirement Specifications}
\author{Group 2} 
\date{December 2025}

\setlength{\droptitle}{6cm}

\pagestyle{fancy}
\fancyhead[L]{JPL - Moon Augmented Reality SRS}
\fancyhead[C]{}
\fancyhead[R]{Page \thepage}

\renewcommand{\UrlFont}{\ttfamily\underline}
\begin{document}

\begin{titlepage}
\maketitle
\thispagestyle{empty}
\end{titlepage}

\thispagestyle{empty}
\tableofcontents
\newpage

\section*{Version History}
\addcontentsline{toc}{section}{Version History}
\begin{table}[ht]
\centering
\begin{tabular}{|c|c|c|}
    \hline
     \textbf{Date} & \textbf{Description} & \textbf{Version} \\
     \hline
     12/4/2025 & Update snapshot 1 & 1.0 \\
     \hline
     12/5/2025 & Update snapshot 2 & 2.0 \\
     \hline
     12/9/2025 & Update snapshot 3 & 3.0 \\
     \hline
     12/11/2025 & Update snapshot 4 & 4.0 \\
     \hline
\end{tabular}
\end{table}

\section{Introduction}
\subsection{Purpose}
The purpose of this Software Requirement Specification (SRS) is to define the functional and non-functional requirements of the Moon Augmented Reality system. It ensures that all stakeholders have a clear understanding of the system’s objectives, capabilities, and constraints before development begins.

\subsection{Intended Audience}
This document is intended for project stakeholders including software developers, project advisors, testers, and Jet Propulsion Laboratory (JPL) collaborators. It may also be useful for future teams who will maintain or extend the system.

\subsection{Overview of the Software}
The Moon Augmented Reality system allows users to upload telescope images of the Moon and receive augmented overlays highlighting lunar features such as craters, maria, and landing sites. It integrates JPL’s MoonTrek datasets with user-provided images to generate interactive visualizations. The system also aims to improve image registration accuracy, support telescope-computer communication, and generate 3D models of the Moon, Earth, and Sun. These models will annotate the location and time of image capture.

\section{External Interface Requirements}
\subsection{User Interface}
The system provides a web-based interface built with Vue.js. Users can upload lunar images, view augmented overlays of lunar features, and interact with a 3D model of the Moon generated from JPL datasets. The interface is responsive and accessible across desktop and mobile devices.

\subsection{Software Interfaces}
The system communicates with NASA’s MoonTrek API for lunar datasets, internal databases for storing user-uploaded images and metadata, and RESTful APIs to handle client-server communication. It also uses Express.js for backend operations and MySQL for data storage.

\section{Legal and Ethical Considerations}
\subsection{Data Storage and Privacy Considerations}
User-uploaded images are stored securely with encryption. Metadata such as upload time and telescope type may be collected, but no personal identifiers are stored. Users retain ownership of their images. The system complies with data protection standards and ensures secure access to stored content.

\subsection{Legal and Ethical Issues}
The system complies with NASA’s data-sharing policies and ensures user consent for uploaded images. Ethical considerations include protecting user privacy, preventing misuse of lunar data, and ensuring transparency in how data is processed. The system avoids collecting sensitive personal data and provides clear terms of use.

\newpage

\section*{Glossary}
\addcontentsline{toc}{section}{Glossary}
\begin{table}[H]
\centering
\renewcommand{\arraystretch}{1.4}
\begin{tabular}{|c|c|}
\hline
\textbf{Acronym} & \textbf{Long Version} \\
\hline
SRS & Software Requirement Specification \\
\hline
UI & User Interface \\
\hline
API & Application Programming Interface \\
\hline
DB & Database \\
\hline
AR & Augmented Reality \\
\hline
JPL & Jet Propulsion Laboratory \\
\hline
MT & Moon Trek \\
\hline
\end{tabular}
\end{table}


\section*{References}
\addcontentsline{toc}{section}{References}
\begin{table}[h!]
\centering
\renewcommand{\arraystretch}{1.6}
\begin{tabular}{|p{6cm}|p{11cm}|}
\hline
\textbf{Reference Name} & \textbf{Source} \\
\hline
Software Requirement Specification Template & CSULA \\
\hline
MoonTrek API & \url{https://trek.nasa.gov/tiles/apidoc/trekAPI.html?bod} \\
\hline
Vue.js Documentation & \url{https://vuejs.org/guide/introduction.html} \\
\hline
Express.js Documentation & \url{https://expressjs.com/en/starter/installing.html} \\
\hline
MySQL Workbench & \url{https://dev.mysql.com/doc/workbench/en/} \\
\hline
\end{tabular}
\end{table}

\end{document}