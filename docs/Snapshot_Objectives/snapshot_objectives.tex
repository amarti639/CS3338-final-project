\documentclass[12pt]{article}
\usepackage{geometry}
\usepackage{float}
\usepackage{fancyhdr}
\usepackage{graphicx}
\usepackage{titling}
\usepackage{hyperref}
\usepackage[hyphens]{url}
\usepackage[normalem]{ulem}

\geometry{a4paper, margin=1in}
\Urlmuskip=0mu plus 1mu

\title{JPL - Moon Augmented Reality \\ Snapshot Objectives}
\author{Group 2} 
\date{December 2025}

\setlength{\droptitle}{6cm}

\pagestyle{fancy}
\fancyhead[L]{Snapshot Objectives}
\fancyhead[C]{}
\fancyhead[R]{Page \thepage}

\begin{document}

\begin{titlepage}
\maketitle
\thispagestyle{empty}
\end{titlepage}

\section*{Start Objective}

The starting objective of the MoonTrek Augmented Reality Application is to establish the user interface and backend components necessary for the web application. This includes the following key components:

\begin{itemize}
    \item Setting up the Docker environment with Vue.js frontend, Express.js backend, MySQL database, and Python image processing service.
    \item Developing a user-friendly interface on the Vue.js web application with navigation between Home, Upload, and Model pages.
    \item Implementing the MySQL database to store user images and EXIF metadata (timestamp and GPS coordinates).
    \item Creating image upload functionality using Multer middleware to handle JPEG and PNG files up to 10MB.
    \item Extracting metadata from uploaded images automatically, with a fallback form for manual entry when metadata is missing.
    \item Establishing connection to NASA MoonTrek API and documenting available endpoints for lunar feature data.
\end{itemize}

\section*{1st Checkpoint Objective}

The primary objective for Snapshot 2 is to implement the core image processing algorithms and 3D model generation for context-aware image registration. This includes:

\begin{itemize}
    \item Developing the circle detection algorithm using Python and OpenCV to automatically identify the moon in uploaded images.
    \item Implementing the SIFT (Scale-Invariant Feature Transform) algorithm for image registration between user images and reference images.
    \item Building the Three.js 3D model that positions Earth, Moon, and Sun based on the user's timestamp and GPS coordinates.
    \item Applying LRO WAC texture mapping to the 3D moon model and implementing directional lighting for realistic shadows.
    \item Generating reference images from the 3D model that match the user's viewing angle for accurate SIFT registration.
    \item Calculating transformation matrices to account for rotation, scale, and translation differences between images.
\end{itemize}

\section*{2nd Checkpoint Objective}

The primary objective for Snapshot 3 is to integrate the NASA MoonTrek API and implement the overlay system that displays lunar features on user images. This includes:

\begin{itemize}
    \item Integrating NASA MoonTrek API to fetch crater names, maria (lunar seas), and Apollo landing site data.
    \item Implementing coordinate mapping to translate WAC composite coordinates to user image pixel positions using the transformation matrix.
    \item Building the overlay display system that accurately positions crater names and landing site markers on the annotated image.
    \item Creating interactive UI controls including a dropdown menu for layer selection and an opacity slider for overlay transparency.
    \item Optimizing the processing pipeline to ensure the complete workflow from upload to annotated result completes within 60 seconds.
\end{itemize}

\section*{Due Date Checkpoint}

The primary objective of Snapshot 4, or the Due Date checkpoint, is to finalize the YourMoon database feature and make final adjustments to the application. This includes:

\begin{itemize}
    \item Implementing the YourMoon database submission system with an image cropping interface for users to contribute moon images.
    \item Integrating machine learning validation to verify submitted images are actual moon photographs.
    \item Conducting comprehensive testing across different browsers (Chrome, Firefox, Safari, Edge) to ensure WebGL compatibility.
    \item Performing final optimizations to improve SIFT registration speed and overall system performance.
    \item Completing all documentation including final versions of the SDD, SRS, Design Specification, and README user manual.
    \item Making minor interface adjustments to improve usability and address any remaining bugs or errors discovered during testing.
\end{itemize}

\end{document}